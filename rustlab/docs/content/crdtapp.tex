% block diagram
% explain each component
% brief about CRDTs
% why use CRDTs (example)
% explain structs and interfaces in crdt.rs
% explain WebSocket server endpoints

\subsection{System Design}

The \verb|WebSocket| based approach implements a \verb|WebSocket| enabled server which handles incoming requests from clients and promotes them to a \verb|ws| connection which is handled by a different thread using \verb|tokio::spawn()|. 

\begin{itemize}
    \item \textbf{crdt/} - Contains \verb|struct|s and datatypes for client-server communication, storing the sheet data model, and CRDT (Conflict-free replicating data types)-based \verb|struct|s.
    \item \textbf{server/} - Contains implementation of a simple WebSocket based server which handles \verb|on_connection|, \verb|grid_update|, and \verb|on_close| requests from the client. Uses \verb|tokio| and \verb|tokio::tungstenite|.
    \item \textbf{ws\_client/} - A WebSocket-enabled client using \verb|Leptos| and uses timestamps to update and make appropriate requests to the server.
\end{itemize}

\subsection{Structs and Interfaces}

\subsubsection{\textbf{crdt/}}

\begin{itemize}
    \item \textbf{Client} - represents a Client with a unique name at the server.
    \item \textbf{Event} - represents a generic event with some data and a typef
    \item \textbf{InitEvent} - represents an \verb|Init| event when client wants to initiate a connection. The \verb|InitEvent| contains a name string.
    \item \textbf{GridUpdateEvent} - represents a grid update event when a client makes a change to the sheet. It contains the name of the change-maker and the entire grid data.
    \item \textbf{ClientListEvent} - represents the list of clients currently connected to the server. The server broadcasts this information for every client to update its own client list.
    \item \textbf{Column} - represents a single cell. It contains the name of the last client who changed it, the timestamp, the index, and the value.
    \item \textbf{Row} - represents a \verb|Row| in the sheet. Each \verb|Row| contains a list of \verb|Column| structs.
\end{itemize}

\subsubsection{\textbf{server/}}

The \verb|server| object is a simple WebSocket-based server which listens on port \verb|3030| for any client connection requests. It spawns a thread for each client and goes into an infinite loop serving that client until the client disconnects. There are separate handlers for each client action: 
\begin{itemize}
    \item \verb|accept_connection(TcpStream, Clients)| - using \verb|tokio::tungstenite|, we promote the connection to a WS connection and match the client request to event type and dispatch the corresponding handler.
    \item \verb|handle_init(InitEvent, Clients)| - adds this client to the \verb|ClientList| struct and broadcasts the list so that other clients update their local lists.
    \item \verb|handle_grid_update(GridUpdateEvent, Clients)| - simply read, decode the Event object and broadcast to everyone on the client list.
    \item \verb|handle_close(client_id, Clients)| - remove this client from the \verb|ClientList| and broadcast the updated \verb|ClientList| to everyone. 
\end{itemize}

\subsubsection{\textbf{client/}}

The \verb|ws_client| module uses \verb|Leptos| to build a simple frontend for the application. The markup is stored in \verb|index.html| which contains an input field for the username, a Client list to show in the DOM and a table component for rendering the grid.

The \verb|App| component in \verb|ws_client/lib.rs| uses two effects to handle local state from incoming server messages and another to propagate its own local state to the server using client requests.

The rest of the code builds the following components:

\begin{itemize}
    \item \textbf{App} - main component which renders the page and handles server communication. It also handles changes to the local sheet data if required.
    \item \textbf{Connect} - a simple \verb|FormInput| component which sends an \verb|InitEvent| request to the server.
    \item \textbf{Clients} - a list of actively connected clients as broadcasted by the server in its \verb|ClientListEvent| message.
    \item \textbf{Grid} - a table component which renders the spreadsheet and sends a \verb|GridUpdateEvent| to the server if any of the cell is updated by the user.
\end{itemize}

\subsection{Challenges faced}

We faced multiple challenges during the implementation of this app. Some of them are listed below:

\begin{itemize}
    \item \textbf{Lack of accessible literature on CRDT} - This made it difficult to understand CRDTs as a beginner from reading academic papers. Eventually, we found few blogposts which explained the concept well.
    \item \textbf{Understanding tokio library's async/await model} - The async/await model used by \verb|tokio| to implement its server was not clear to us and took time to implement properly to handle multiple client requests. 
    \item \textbf{Outdated Leptos packages} - There were few outdated \verb|leptos-use| package which we had to use for constraints with \verb|leptos|. This made us rewrite the \verb|ws_client| using more primitive \verb|web_sys::WebSocket|, which Leptos depends on.
\end{itemize}
