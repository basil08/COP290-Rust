% \documentclass[12pt]{article}
% \usepackage[margin=1in]{geometry}
% \usepackage{enumitem}
% \usepackage{hyperref}
% \usepackage{fancyhdr}
% \pagestyle{fancy}
% \fancyhf{}
% \rhead{Spreadsheet Project Report}
% \lhead{COP290 - IIT Delhi}
% \rfoot{\thepage}

\title{Spreadsheet System: Design, Extensions, and Architecture Report}

\subsection*{2.1. Command-Line Interface and Mode Selection}
Our Rust-based spreadsheet system supports two runtime-selectable modes: \textbf{Standard} and \textbf{Extended}. This design enables clean separation of concerns and feature modularity.

\begin{itemize}[leftmargin=*]
    \item \textbf{Standard Mode:}
    \begin{verbatim}
    make 
    ./target/release/spreadsheet 10 10
    \end{verbatim}
    \textit{Modules Used:} \texttt{parser.rs}, \texttt{function.rs}, \texttt{graph.rs}, \texttt{display.rs}

    \textit{Core Features:}
    \begin{itemize}
        \item Integer-only cell support
        \item Basic arithmetic operations (+, -, *, /)
        \item Range functions: SUM, AVG, MIN, MAX, STDEV
        \item Sleep functionality to delay cell evaluation
    \end{itemize}

    \item \textbf{Extended Mode:}
    \begin{verbatim}
    make 
    ./target/release/spreadsheet -extended 10 10
    \end{verbatim}
    \textit{Modules Used:} \texttt{parser\_ext.rs}, \texttt{function\_ext.rs}, \texttt{graph\_ext.rs}, \texttt{display\_ext.rs}, \texttt{util\_ext.rs}

    \textit{Additional Features:}
    \begin{itemize}
        \item Support for float and string cell types
        \item String concatenation using '+'
        \item Pattern-based Autofill: Arithmetic (AP), Geometric (GP), Fibonacci, Constant
        \item Undo/Redo via \texttt{StateSnapshot}
    \end{itemize}
\end{itemize}

\subsection*{2.2. Software Architecture}
The system follows a layered, modular architecture:

\textbf{Core Modules:}
\begin{itemize}
    \item \texttt{parser.rs} / \texttt{parser\_ext.rs} – expression parsing
    \item \texttt{function.rs} / \texttt{function\_ext.rs} – function evaluations
    \item \texttt{graph.rs} / \texttt{graph\_ext.rs} – dependency tracking and cycle detection
    \item \texttt{display.rs} / \texttt{display\_ext.rs} – terminal UI rendering
    \item \texttt{util\_ext.rs} – shared utilities (e.g., polymorphic arithmetic handling)
\end{itemize}

\textbf{Entry Point:} \texttt{rustlab/cli/main.rs} dynamically selects between standard and extended based on CLI flag \texttt{-extended}.

\subsection*{2.3. Key Extensions Implemented}
\begin{itemize}
    \item \textbf{Typed Cells:} Cells can now hold \texttt{Int}, \texttt{Float}, or \texttt{String} using an enum-based \texttt{CellValue}.
    \item \textbf{String Operations:} Support for string assignment and `+`-based concatenation.
    \item \textbf{Mixed-Type Arithmetic:} Fully supports operations like `Float + Int`, `Int / Int`, and errors on invalid ops (e.g., `String - Float`).
    \item \textbf{Autofill Feature:} From a 4-cell seed, detects AP/GP/Fibonacci/Constant and populates the column up to a given length.
    \item \textbf{Undo/Redo:} Snapshots captured as \texttt{StateSnapshot}, used for user-directed rollback.
\end{itemize}

\subsection*{2.4. Primary Data Structures}
\begin{itemize}
    \item \texttt{CellValue (enum)}: \texttt{Int(i32) | Float(f64) | String(String)}
    \item \texttt{Cell}: Holds \texttt{CellValue} and a validity flag
    \item \texttt{Formula}: Stores op\_type, op\_info1, op\_info2 to describe each cell’s formula
    \item \texttt{Graph / GraphExt}: Adjacency list and range dependency representation
    \item \texttt{State / StateSnapshot}: Captures pre-modification state for undo/redo
\end{itemize}

\subsection*{2.5. Module Interfaces}
\begin{itemize}
    \item \texttt{parser → graph}: Adds/removes edges and ranges
    \item \texttt{parser → function}: Invokes range or arithmetic evaluation
    \item \texttt{graph → recalc}: Handles topological sort and propagation
    \item \texttt{display → core}: Renders spreadsheet with ERR/value handling
    \item \texttt{parser\_ext → util\_ext}: Performs typed arithmetic via \texttt{arithmetic\_eval}
\end{itemize}

\subsection*{2.6. Encapsulation Strategies}
\begin{itemize}
    \item Separate file namespace for `\_ext` modules; standard and extended logic never mix
    \item Use of stateless pure functions where possible
    \item Unsafe usage (\texttt{static mut}) restricted to well-documented parser.rs
    \item Public APIs shield internals of evaluation and recalculation logic
\end{itemize}

\subsection*{2.7. Design Justification}
\begin{itemize}
    \item Easy switching between standard and extended logic based on mode
    \item Clear path for future extensions (e.g., Date type, Graph plots)
    \item Minimized duplication through shared utilities like \texttt{util\_ext.rs}
    \item Debug-friendly separation: extended features can be tested without affecting standard ones
\end{itemize}

\subsection*{2.8. Design Modifications During Development}
\begin{itemize}
    \item Introduced \texttt{static mut HAS\_CYCLE} and \texttt{INVALID\_RANGE} for backtracking in parser
    \item Moved from shared modules to a clean \texttt{\_ext.rs} hierarchy
    \item Added pattern recognition + generation to support autofill logic
    \item Modified formula structure to accommodate multi-typed cells
\end{itemize}



\subsection*{2.9. Demonstration of Extensions}

The following session demonstrates the extended features including string operations, float handling, mixed-type formula evaluation, and autofill.

\subsubsection*{Running the Spreadsheet in Extended Mode}

\begin{verbatim}
$ ./target/release/spreadsheet -extended 10 10
\end{verbatim}

This command launches a 10x10 spreadsheet grid in extended mode.

\subsubsection*{Testing String Cell Support}

We input string values and concatenate them:

\begin{verbatim}
> A1="hi"
> A2="hello"
> A3=A1+A2
\end{verbatim}

Result:
\begin{itemize}
  \item A1 = "hi"
  \item A2 = "hello"
  \item A3 = "hihello" (concatenation successful)
\end{itemize}
If A1 is changed to "hello", A3 will be changed to "hellohello"
\subsubsection*{Testing Float Support and Arithmetic}

We input float and integer values, then add them:

\begin{verbatim}
> B1=1.09
> B2=90
> B3=B1+B2
\end{verbatim}

Result:
\begin{itemize}
  \item B1 = 1.09
  \item B2 = 90
  \item B3 = 91.09
\end{itemize}
If B2 is changed to 0, B3 will be changed to 1.09
\subsubsection*{Autofill Feature Demonstration}

We manually seed a pattern and autofill a column:

\begin{verbatim}
> C1=1
> C2=2
> C3=3
> C4=4
> =autofill C 10
\end{verbatim}

Result:
\begin{itemize}
  \item Column C is autofilled as 1, 2, 3, 4, 5, 6, 7, 8, 9, 10.
\end{itemize}

Similarly, patterns like AP, GP, fibonacci and constants can be autofilled.

\subsubsection*{Undo/Redo Functionality }
Demonstrates snapshot-based state restoration:
\begin{verbatim}
> A1=90
> A2=8
> undo        % Reverts A2=8
> undo        % Reverts A1=90
> redo        % Redoes A1=90
> redo        % Redoes A2=8
> redo        % No effect (nothing to redo)
\end{verbatim}
    Result:
\begin{itemize}
    \item After first undo: A1 = 90, A2 = 0
    \item After second undo: A1 = 0, A2 = 0
    \item After two redos: A1 = 90, A2 = 8
\end{itemize}


